En esta sección se presentan los índices y triggers implementados para mejorar el rendimiento de consultas frecuentes y asegurar restricciones de negocio directamente en la base de datos.

\subsection{Índices implementados}

Se definieron los siguientes índices secundarios para optimizar el acceso a los datos mediante claves foráneas o combinaciones frecuentes de columnas.

\begin{lstlisting}
CREATE INDEX idx_producto_categoria ON Producto(id_categoria);
CREATE INDEX idx_pedido_cliente ON Pedido(id_cliente);
CREATE INDEX idx_detalle_pedido_producto ON DetallePedido(id_producto);
CREATE INDEX idx_resena_cliente_producto ON Resena(id_cliente, id_producto);
\end{lstlisting}

\textbf{Explicación de cada índice:}

\begin{itemize}
  \item \texttt{idx\_producto\_categoria}  
  Mejora las consultas que recuperan productos por categoría, como:  
  \texttt{SELECT * FROM Producto WHERE id\_categoria = 3;}
  
  \item \texttt{idx\_pedido\_cliente}  
  Optimiza las búsquedas de pedidos realizados por un cliente:  
  \texttt{SELECT * FROM Pedido WHERE id\_cliente = 12;}
  
  \item \texttt{idx\_detalle\_pedido\_producto}  
  Acelera la identificación de pedidos que incluyen un producto específico:  
  \texttt{SELECT * FROM DetallePedido WHERE id\_producto = 5;}
  
  \item \texttt{idx\_resena\_cliente\_producto}  
  Este índice compuesto facilita consultas como:  
  \texttt{SELECT * FROM Reseña WHERE id\_cliente = 1 AND id\_producto = 10;}  
  lo cual es útil para validar si un cliente ya reseñó un producto.
\end{itemize}

\subsection{Trigger: Límite de 5 pedidos pendientes por cliente}

Para restringir la cantidad de pedidos pendientes a un máximo de cinco por cliente, se implementó el siguiente trigger:

\begin{lstlisting}
DELIMITER //

CREATE TRIGGER trg_max_pedidos_pendientes
BEFORE INSERT ON Pedido
FOR EACH ROW
BEGIN
  DECLARE pedidos_pendientes INT;

  SELECT COUNT(*) INTO pedidos_pendientes
  FROM Pedido
  WHERE id_cliente = NEW.id_cliente AND estado = 'pendiente';

  IF pedidos_pendientes >= 5 THEN
    SIGNAL SQLSTATE '45000'
    SET MESSAGE_TEXT = 'El cliente ya tiene 5 pedidos pendientes.';
  END IF;
END;
//

DELIMITER ;
\end{lstlisting}

\textbf{Explicación:}

\begin{itemize}
  \item El trigger se ejecuta antes de insertar un nuevo pedido.
  \item Cuenta los pedidos del cliente con estado \texttt{pendiente}.
  \item Si el número es igual o mayor a 5, se lanza una excepción y se bloquea la inserción.
  \item Esto garantiza que se cumpla la restricción incluso fuera del sistema de interfaz (por ejemplo, por scripts o inyecciones directas).
\end{itemize}

\subsection{Trigger: Validar reseñas solo de clientes que hayan comprado el producto}

Este trigger impide que un cliente deje una reseña sobre un producto si no lo ha comprado previamente. Así se protege la integridad del sistema de calificaciones.

\begin{lstlisting}
DELIMITER //

CREATE TRIGGER trg_resena_solo_si_compro
BEFORE INSERT ON Resena
FOR EACH ROW
BEGIN
  DECLARE total_compras INT;

  SELECT COUNT(*) INTO total_compras
  FROM Pedido P
  JOIN DetallePedido D ON P.id_pedido = D.id_pedido
  WHERE P.id_cliente = NEW.id_cliente
    AND D.id_producto = NEW.id_producto;

  IF total_compras = 0 THEN
    SIGNAL SQLSTATE '45000'
    SET MESSAGE_TEXT = 'El cliente no ha comprado este producto.';
  END IF;
END;
//

DELIMITER ;
\end{lstlisting}

\textbf{Explicación:}

\begin{itemize}
  \item El trigger se ejecuta antes de insertar una fila en la tabla \texttt{Reseña}.
  \item Se hace una consulta a las tablas \texttt{Pedido} y \texttt{DetallePedido} para verificar si el cliente efectivamente compró el producto.
  \item Si el resultado es cero, significa que el cliente no lo ha comprado.
  \item En ese caso, el trigger lanza un error personalizado que bloquea la inserción de la reseña.
\end{itemize}

Este control asegura que las reseñas provienen exclusivamente de clientes reales, reforzando la autenticidad de la retroalimentación.
