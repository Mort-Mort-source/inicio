

\definecolor{codebg}{rgb}{0.95,0.95,0.95}
\lstset{
    backgroundcolor=\color{codebg},
    basicstyle=\ttfamily\small,
    breakatwhitespace=false,
    breaklines=true,
    captionpos=b,
    frame=single,
    keepspaces=true,
    showspaces=false,
    showstringspaces=false,
    showtabs=false,
    tabsize=4,
    language=SQL
}



\section{Implementación en SQL}

\subsection{Creación de la base de datos y tablas}

\begin{lstlisting}
CREATE DATABASE EMPRESA;
USE EMPRESA;
\end{lstlisting}

\subsection{Tabla Empleados}

\begin{lstlisting}
CREATE TABLE Empleados (
    id INT AUTO_INCREMENT PRIMARY KEY,
    nombre VARCHAR(100),
    fecha_de_ingreso DATE,
    salario DECIMAL(10,2),
    categoria VARCHAR(100)
);
\end{lstlisting}

\textbf{Descripción de campos:}
\begin{itemize}
    \item \texttt{id}: Identificador único del empleado (clave primaria)
    \item \texttt{nombre}: Nombre completo del empleado
    \item \texttt{fecha\_de\_ingreso}: Fecha de contratación
    \item \texttt{salario}: Remuneración del empleado
    \item \texttt{categoria}: Categoría o puesto del empleado
\end{itemize}

\subsection{Tabla Tabulador}

\begin{lstlisting}
CREATE TABLE Tabulador (
    id INT AUTO_INCREMENT PRIMARY KEY,
    nombre_puesto VARCHAR(100),
    sueldo DECIMAL(10,2),
    dias_antiguedad INT
);
\end{lstlisting}

\textbf{Descripción de campos:}
\begin{itemize}
    \item \texttt{id}: Identificador único del registro (clave primaria)
    \item \texttt{nombre\_puesto}: Nombre del puesto o categoría
    \item \texttt{sueldo}: Salario base para el puesto
    \item \texttt{dias\_antiguedad}: Días requeridos para antigüedad en el puesto
\end{itemize}

\end{document}