
Este capitulo presenta la implementacion del modelo entidad-relacion en el sistema de gestion de bases de datos MySQL. Se creo una base de datos llamada \texttt{TiendaOnline} y a continuacion se describen las tablas que la conforman.

\subsection{Tabla Categoria}

La tabla \texttt{Categoria} organiza los productos en grupos tematicos como telefonos, laptops o accesorios. Contiene una clave primaria que la identifica de forma unica.

\begin{lstlisting}
CREATE TABLE Categoria (
  id_categoria INT AUTO_INCREMENT PRIMARY KEY,
  nombre VARCHAR(100) NOT NULL,
  descripcion TEXT
);
\end{lstlisting}

\textbf{Descripcion de campos:}

\begin{itemize}
  \item \texttt{id\_categoria}: Identificador unico de la categoria.
  \item \texttt{nombre}: Nombre de la categoria (por ejemplo, "Laptops").
  \item \texttt{descripcion}: Descripcion opcional de la categoria.
\end{itemize}

\subsection{Tabla Producto}

Almacena los productos disponibles en la tienda. Cada producto pertenece a una categoria.

\begin{lstlisting}
CREATE TABLE Producto (
  id_producto INT AUTO_INCREMENT PRIMARY KEY,
  nombre VARCHAR(100) NOT NULL,
  descripcion TEXT,
  precio DECIMAL(10,2) NOT NULL,
  stock INT NOT NULL CHECK (stock >= 0),
  id_categoria INT,
  FOREIGN KEY (id_categoria) REFERENCES Categoria(id_categoria)
);
\end{lstlisting}

\textbf{Descripcion de campos:}

\begin{itemize}
  \item \texttt{id\_producto}: Identificador del producto.
  \item \texttt{nombre}, \texttt{descripcion}, \texttt{precio}: Informacion basica.
  \item \texttt{stock}: Inventario disponible, no puede ser negativo.
  \item \texttt{id\_categoria}: Clave foranea que enlaza con la tabla \texttt{Categoria}.
\end{itemize}

\subsection{Tabla Cliente}

Contiene la informacion personal de cada cliente. El correo electronico debe ser unico.

\begin{lstlisting}
CREATE TABLE Cliente (
  id_cliente INT AUTO_INCREMENT PRIMARY KEY,
  nombre VARCHAR(100) NOT NULL,
  correo VARCHAR(100) NOT NULL UNIQUE,
  direccion TEXT,
  telefono VARCHAR(20)
);
\end{lstlisting}

\textbf{Descripcion de campos:}

\begin{itemize}
  \item \texttt{id\_cliente}: Identificador del cliente.
  \item \texttt{correo}: Sirve como clave candidata por ser unico.
  \item \texttt{direccion}, \texttt{telefono}: Informacion de contacto.
\end{itemize}

\subsection{Tabla Pedido}

Almacena las ordenes hechas por los clientes, incluyendo la fecha y el estado del pedido.

\begin{lstlisting}
CREATE TABLE Pedido (
  id_pedido INT AUTO_INCREMENT PRIMARY KEY,
  fecha DATE NOT NULL,
  estado ENUM('pendiente', 'enviado', 'entregado') NOT NULL,
  id_cliente INT,
  FOREIGN KEY (id_cliente) REFERENCES Cliente(id_cliente)
);
\end{lstlisting}

\textbf{Descripcion de campos:}

\begin{itemize}
  \item \texttt{id\_pedido}: Clave primaria del pedido.
  \item \texttt{fecha}: Fecha en que se realizo.
  \item \texttt{estado}: Estado actual del pedido (controlado por ENUM).
  \item \texttt{id\_cliente}: Clave foranea hacia el cliente que realizo el pedido.
\end{itemize}

\subsection{Tabla DetallePedido}

Esta tabla representa una relacion \texttt{muchos a muchos} entre \texttt{Pedido} y \texttt{Producto}. Registra los productos incluidos en cada pedido, sus cantidades y precios al momento de la compra.

\begin{lstlisting}
CREATE TABLE DetallePedido (
  id_detalle INT AUTO_INCREMENT PRIMARY KEY,
  id_pedido INT,
  id_producto INT,
  cantidad INT NOT NULL,
  precio_unitario DECIMAL(10,2) NOT NULL,
  FOREIGN KEY (id_pedido) REFERENCES Pedido(id_pedido),
  FOREIGN KEY (id_producto) REFERENCES Producto(id_producto)
);
\end{lstlisting}

\textbf{Descripcion detallada:}

\begin{itemize}
  \item \texttt{id\_detalle}: Identificador unico de cada linea del pedido.
  \item \texttt{id\_pedido}: Clave foranea hacia el pedido correspondiente.
  \item \texttt{id\_producto}: Clave foranea hacia el producto que se esta comprando.
  \item \texttt{cantidad}: Numero de unidades del producto en ese pedido.
  \item \texttt{precio\_unitario}: Precio del producto en el momento del pedido (puede variar del precio actual).
\end{itemize}

Esta tabla es esencial para conservar un historico detallado de cada compra, permitiendo consultas como "¿que productos se compraron en el pedido 123?" o "¿cuantas unidades de producto A se vendieron en total?".

\subsection{Tabla Resena}

Permite registrar las calificaciones y comentarios hechos por los clientes sobre los productos comprados.

\begin{lstlisting}
CREATE TABLE Resena (
  id_resena INT AUTO_INCREMENT PRIMARY KEY,
  calificacion INT CHECK (calificacion BETWEEN 1 AND 5),
  comentario TEXT,
  id_producto INT,
  id_cliente INT,
  FOREIGN KEY (id_producto) REFERENCES Producto(id_producto),
  FOREIGN KEY (id_cliente) REFERENCES Cliente(id_cliente)
);
\end{lstlisting}

\textbf{Descripcion de campos:}

\begin{itemize}
  \item \texttt{id\_resena}: Identificador de la resena.
  \item \texttt{calificacion}: Numero entero entre 1 y 5.
  \item \texttt{comentario}: Texto libre.
  \item \texttt{id\_producto}, \texttt{id\_cliente}: Claves foraneas para validar que el cliente resena un producto especifico.
\end{itemize}

