\documentclass[12pt]{article}
\usepackage[utf8]{inputenc}
\usepackage[T1]{fontenc}
\usepackage[spanish]{babel}
\usepackage{graphicx}
\usepackage{geometry}
\usepackage{titlesec}
\usepackage{setspace}
\usepackage{hyperref}
\usepackage{enumitem}
\usepackage{listings}
\usepackage{xcolor}
\usepackage{framed}

\geometry{letterpaper, margin=2.5cm}
\setlength{\parskip}{1em}
\setlength{\parindent}{0pt}
\titleformat{\section}{\large\bfseries}{\thesection}{1em}{}
\titleformat{\subsection}{\normalsize\bfseries}{\thesubsection}{1em}{}

\hypersetup{
    colorlinks=true,
    linkcolor=blue,
    urlcolor=blue,
    pdftitle={Tienda Online de Productos Electrónicos},
    pdfauthor={Aaron Rodrigo Ramos Reyes}
}

\begin{document}

% potada
\begin{titlepage}
    \centering
    \vspace*{2cm}
    
    {\Huge\bfseries Proyecto Final Tienda Online de Productos Electrónicos \par}
    \vspace{2cm}
    
   
    
    {\large Alumno: \textbf{Aaron Rodrigo Ramos Reyes} \par}
    {\large Profesor: \textbf{Guillermo Monroy Rodríguez} \par}
    {\large Materia: \textbf{Bases de Datos} \par}
    {\large Universidad Autónoma Metropolitana \par}
    {\large Unidad Cuajimalpa \par}
    \vspace{1cm}
    
    {\large \today \par}
\end{titlepage}


% Índice
\tableofcontents
\newpage

\section{Introducción}-
\subsection{Introducción al Proyecto}

El proyecto consiste en el diseño e implementación de una base de datos en \textbf{MySQL}, la cual debe cumplir con todos los requisitos descritos en el documento \textit{"ProyectoFinalBD"}. 

Con este propósito, se requerirá el uso y reforzamiento de tecnologías como \textbf{PlantUML} para la elaboración de diagramas, \LaTeX{} para generar documentación clara, estructurada y modificable, y \textbf{MySQL} para la implementación de la base de datos. Asimismo, se utilizarán otras herramientas como los \textit{stored procedures} para extender la funcionalidad de la base de datos.

Adicionalmente, se plantea la posibilidad de desarrollar un sistema en \textbf{Java} utilizando una arquitectura de tres capas, lo cual será opcional y dependerá del tiempo disponible para la entrega. Esta etapa adicional busca integrar y aplicar conocimientos adquiridos en otros cursos del plan de estudios.


% asi se hacen las secciones
\section{Diagrama ER y justificación de la normalización}
% hasta aca

%asi se hacen las subsecciones
\subsection{Diagrama Entidad-Relación}
\input{diagramaER}
%hasta aca

\subsection{Justificación de la normalización}
El modelo ha sido diseñado cumpliendo con las tres primeras formas normales para asegurar consistencia, evitar redundancia y facilitar el mantenimiento de los datos:

\begin{itemize}[label=--]
    \item \textbf{Primera Forma Normal (1FN):} Todas las columnas contienen valores atómicos y no hay grupos repetitivos.
    \item \textbf{Segunda Forma Normal (2FN):} No existen dependencias parciales; todos los atributos no clave dependen completamente de la clave primaria.
    \item \textbf{Tercera Forma Normal (3FN):} No hay dependencias transitivas; cada atributo no clave depende solamente de la clave primaria.
\end{itemize}



\section{Implementación en MySQL}


\definecolor{codebg}{rgb}{0.95,0.95,0.95}
\lstset{
    backgroundcolor=\color{codebg},
    basicstyle=\ttfamily\small,
    breakatwhitespace=false,
    breaklines=true,
    captionpos=b,
    frame=single,
    keepspaces=true,
    showspaces=false,
    showstringspaces=false,
    showtabs=false,
    tabsize=4,
    language=SQL
}



\section{Implementación en SQL}

\subsection{Creación de la base de datos y tablas}

\begin{lstlisting}
CREATE DATABASE EMPRESA;
USE EMPRESA;
\end{lstlisting}

\subsection{Tabla Empleados}

\begin{lstlisting}
CREATE TABLE Empleados (
    id INT AUTO_INCREMENT PRIMARY KEY,
    nombre VARCHAR(100),
    fecha_de_ingreso DATE,
    salario DECIMAL(10,2),
    categoria VARCHAR(100)
);
\end{lstlisting}

\textbf{Descripción de campos:}
\begin{itemize}
    \item \texttt{id}: Identificador único del empleado (clave primaria)
    \item \texttt{nombre}: Nombre completo del empleado
    \item \texttt{fecha\_de\_ingreso}: Fecha de contratación
    \item \texttt{salario}: Remuneración del empleado
    \item \texttt{categoria}: Categoría o puesto del empleado
\end{itemize}

\subsection{Tabla Tabulador}

\begin{lstlisting}
CREATE TABLE Tabulador (
    id INT AUTO_INCREMENT PRIMARY KEY,
    nombre_puesto VARCHAR(100),
    sueldo DECIMAL(10,2),
    dias_antiguedad INT
);
\end{lstlisting}

\textbf{Descripción de campos:}
\begin{itemize}
    \item \texttt{id}: Identificador único del registro (clave primaria)
    \item \texttt{nombre\_puesto}: Nombre del puesto o categoría
    \item \texttt{sueldo}: Salario base para el puesto
    \item \texttt{dias\_antiguedad}: Días requeridos para antigüedad en el puesto
\end{itemize}

\end{document}

\end{document}
