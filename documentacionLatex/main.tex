\documentclass[12pt]{article}
\usepackage[utf8]{inputenc}
\usepackage[T1]{fontenc}
\usepackage[spanish]{babel}
\usepackage{graphicx}
\usepackage{geometry}
\usepackage{titlesec}
\usepackage{setspace}
\usepackage{hyperref}
\usepackage{enumitem}
\usepackage{listings}
\usepackage{xcolor}
\usepackage{framed}

% Configuración esencial para listings la vaina que permite mostrar código SQL
\definecolor{codebg}{rgb}{0.95,0.95,0.95}
\lstset{
    backgroundcolor=\color{codebg},
    basicstyle=\ttfamily\footnotesize,
    breaklines=true,
    frame=single,
    language=SQL,
    numbers=left,
    numberstyle=\tiny\color{gray},
    captionpos=b
}

% Configuración de página
\geometry{letterpaper, margin=2.5cm}
\setlength{\parskip}{1em}
\setlength{\parindent}{0pt}
\titleformat{\section}{\large\bfseries}{\thesection}{1em}{}
\titleformat{\subsection}{\normalsize\bfseries}{\thesubsection}{1em}{}

% Configuración de hipervínculos
\hypersetup{
    colorlinks=true,
    linkcolor=blue,
    urlcolor=blue,
    pdftitle={Tienda Online de Productos Electrónicos},
    pdfauthor={Aaron Rodrigo Ramos Reyes}
}

\begin{document}

% Portada
\begin{titlepage}
    \centering
    \vspace*{2cm}
    
    {\Huge\bfseries Proyecto Final Tienda Online de Productos Electrónicos \par}
    \vspace{2cm}
    
   
    
    {\large Alumno: \textbf{Aaron Rodrigo Ramos Reyes} \par}
    {\large Profesor: \textbf{Guillermo Monroy Rodríguez} \par}
    {\large Materia: \textbf{Bases de Datos} \par}
    {\large Universidad Autónoma Metropolitana \par}
    {\large Unidad Cuajimalpa \par}
    \vspace{1cm}
    
    {\large \today \par}
\end{titlepage}


% Índice
\tableofcontents
\newpage

\section{Introducción}
\subsection{Introducción al Proyecto}

El proyecto consiste en el diseño e implementación de una base de datos en \textbf{MySQL}, la cual debe cumplir con todos los requisitos descritos en el documento \textit{"ProyectoFinalBD"}. 

Con este propósito, se requerirá el uso y reforzamiento de tecnologías como \textbf{PlantUML} para la elaboración de diagramas, \LaTeX{} para generar documentación clara, estructurada y modificable, y \textbf{MySQL} para la implementación de la base de datos. Asimismo, se utilizarán otras herramientas como los \textit{stored procedures} para extender la funcionalidad de la base de datos.

Adicionalmente, se plantea la posibilidad de desarrollar un sistema en \textbf{Java} utilizando una arquitectura de tres capas, lo cual será opcional y dependerá del tiempo disponible para la entrega. Esta etapa adicional busca integrar y aplicar conocimientos adquiridos en otros cursos del plan de estudios.


\section{Diagrama ER y justificación de la normalización}
\subsection{Diagrama Entidad-Relación}
\input{diagramaER}

\subsection{Justificación de la normalización}
El modelo ha sido diseñado cumpliendo con las tres primeras formas normales para asegurar consistencia, evitar redundancia y facilitar el mantenimiento de los datos:

\begin{itemize}[label=--]
    \item \textbf{Primera Forma Normal (1FN):} Todas las columnas contienen valores atómicos y no hay grupos repetitivos.
    \item \textbf{Segunda Forma Normal (2FN):} No existen dependencias parciales; todos los atributos no clave dependen completamente de la clave primaria.
    \item \textbf{Tercera Forma Normal (3FN):} No hay dependencias transitivas; cada atributo no clave depende solamente de la clave primaria.
\end{itemize}



\section{Implementación de tablas en MySQL}


\definecolor{codebg}{rgb}{0.95,0.95,0.95}
\lstset{
    backgroundcolor=\color{codebg},
    basicstyle=\ttfamily\small,
    breakatwhitespace=false,
    breaklines=true,
    captionpos=b,
    frame=single,
    keepspaces=true,
    showspaces=false,
    showstringspaces=false,
    showtabs=false,
    tabsize=4,
    language=SQL
}



\section{Implementación en SQL}

\subsection{Creación de la base de datos y tablas}

\begin{lstlisting}
CREATE DATABASE EMPRESA;
USE EMPRESA;
\end{lstlisting}

\subsection{Tabla Empleados}

\begin{lstlisting}
CREATE TABLE Empleados (
    id INT AUTO_INCREMENT PRIMARY KEY,
    nombre VARCHAR(100),
    fecha_de_ingreso DATE,
    salario DECIMAL(10,2),
    categoria VARCHAR(100)
);
\end{lstlisting}

\textbf{Descripción de campos:}
\begin{itemize}
    \item \texttt{id}: Identificador único del empleado (clave primaria)
    \item \texttt{nombre}: Nombre completo del empleado
    \item \texttt{fecha\_de\_ingreso}: Fecha de contratación
    \item \texttt{salario}: Remuneración del empleado
    \item \texttt{categoria}: Categoría o puesto del empleado
\end{itemize}

\subsection{Tabla Tabulador}

\begin{lstlisting}
CREATE TABLE Tabulador (
    id INT AUTO_INCREMENT PRIMARY KEY,
    nombre_puesto VARCHAR(100),
    sueldo DECIMAL(10,2),
    dias_antiguedad INT
);
\end{lstlisting}

\textbf{Descripción de campos:}
\begin{itemize}
    \item \texttt{id}: Identificador único del registro (clave primaria)
    \item \texttt{nombre\_puesto}: Nombre del puesto o categoría
    \item \texttt{sueldo}: Salario base para el puesto
    \item \texttt{dias\_antiguedad}: Días requeridos para antigüedad en el puesto
\end{itemize}

\end{document}


\section{Implementación de índices y triggers en MySQL}
En esta sección se presentan los índices y triggers implementados para mejorar el rendimiento de consultas frecuentes y asegurar restricciones de negocio directamente en la base de datos.

\subsection{Índices implementados}

Se definieron los siguientes índices secundarios para optimizar el acceso a los datos mediante claves foráneas o combinaciones frecuentes de columnas.

\begin{lstlisting}
CREATE INDEX idx_producto_categoria ON Producto(id_categoria);
CREATE INDEX idx_pedido_cliente ON Pedido(id_cliente);
CREATE INDEX idx_detalle_pedido_producto ON DetallePedido(id_producto);
CREATE INDEX idx_resena_cliente_producto ON Resena(id_cliente, id_producto);
\end{lstlisting}

\textbf{Explicación de cada índice:}

\begin{itemize}
  \item \texttt{idx\_producto\_categoria}  
  Mejora las consultas que recuperan productos por categoría, como:  
  \texttt{SELECT * FROM Producto WHERE id\_categoria = 3;}
  
  \item \texttt{idx\_pedido\_cliente}  
  Optimiza las búsquedas de pedidos realizados por un cliente:  
  \texttt{SELECT * FROM Pedido WHERE id\_cliente = 12;}
  
  \item \texttt{idx\_detalle\_pedido\_producto}  
  Acelera la identificación de pedidos que incluyen un producto específico:  
  \texttt{SELECT * FROM DetallePedido WHERE id\_producto = 5;}
  
  \item \texttt{idx\_resena\_cliente\_producto}  
  Este índice compuesto facilita consultas como:  
  \texttt{SELECT * FROM Reseña WHERE id\_cliente = 1 AND id\_producto = 10;}  
  lo cual es útil para validar si un cliente ya reseñó un producto.
\end{itemize}

\subsection{Trigger: Límite de 5 pedidos pendientes por cliente}

Para restringir la cantidad de pedidos pendientes a un máximo de cinco por cliente, se implementó el siguiente trigger:

\begin{lstlisting}
DELIMITER //

CREATE TRIGGER trg_max_pedidos_pendientes
BEFORE INSERT ON Pedido
FOR EACH ROW
BEGIN
  DECLARE pedidos_pendientes INT;

  SELECT COUNT(*) INTO pedidos_pendientes
  FROM Pedido
  WHERE id_cliente = NEW.id_cliente AND estado = 'pendiente';

  IF pedidos_pendientes >= 5 THEN
    SIGNAL SQLSTATE '45000'
    SET MESSAGE_TEXT = 'El cliente ya tiene 5 pedidos pendientes.';
  END IF;
END;
//

DELIMITER ;
\end{lstlisting}

\textbf{Explicación:}

\begin{itemize}
  \item El trigger se ejecuta antes de insertar un nuevo pedido.
  \item Cuenta los pedidos del cliente con estado \texttt{pendiente}.
  \item Si el número es igual o mayor a 5, se lanza una excepción y se bloquea la inserción.
  \item Esto garantiza que se cumpla la restricción incluso fuera del sistema de interfaz (por ejemplo, por scripts o inyecciones directas).
\end{itemize}

\subsection{Trigger: Validar reseñas solo de clientes que hayan comprado el producto}

Este trigger impide que un cliente deje una reseña sobre un producto si no lo ha comprado previamente. Así se protege la integridad del sistema de calificaciones.

\begin{lstlisting}
DELIMITER //

CREATE TRIGGER trg_resena_solo_si_compro
BEFORE INSERT ON Resena
FOR EACH ROW
BEGIN
  DECLARE total_compras INT;

  SELECT COUNT(*) INTO total_compras
  FROM Pedido P
  JOIN DetallePedido D ON P.id_pedido = D.id_pedido
  WHERE P.id_cliente = NEW.id_cliente
    AND D.id_producto = NEW.id_producto;

  IF total_compras = 0 THEN
    SIGNAL SQLSTATE '45000'
    SET MESSAGE_TEXT = 'El cliente no ha comprado este producto.';
  END IF;
END;
//

DELIMITER ;
\end{lstlisting}

\textbf{Explicación:}

\begin{itemize}
  \item El trigger se ejecuta antes de insertar una fila en la tabla \texttt{Reseña}.
  \item Se hace una consulta a las tablas \texttt{Pedido} y \texttt{DetallePedido} para verificar si el cliente efectivamente compró el producto.
  \item Si el resultado es cero, significa que el cliente no lo ha comprado.
  \item En ese caso, el trigger lanza un error personalizado que bloquea la inserción de la reseña.
\end{itemize}

Este control asegura que las reseñas provienen exclusivamente de clientes reales, reforzando la autenticidad de la retroalimentación.


\section{Consultas de almacenamiento en MySQL}
En esta sección se presentan tres procedimientos almacenados implementados en MySQL que permiten realizar consultas comunes dentro del sistema de gestión de una tienda en línea. Cada procedimiento tiene una finalidad específica y fue optimizado para consultar múltiples tablas relacionadas.

\subsection{Productos disponibles por categoría, ordenados por precio}

Este procedimiento lista todos los productos disponibles (con stock mayor a cero), agrupados por categoría y ordenados por precio en orden ascendente.

\begin{lstlisting}
DELIMITER //

CREATE PROCEDURE ProductosPorCategoria()
BEGIN
  SELECT 
    c.nombre AS categoria,
    p.nombre AS producto,
    p.descripcion,
    p.precio,
    p.stock
  FROM Producto p
  JOIN Categoria c ON p.id_categoria = c.id_categoria
  WHERE p.stock > 0
  ORDER BY c.nombre, p.precio ASC;
END;
//

DELIMITER ;
\end{lstlisting}

\textbf{Explicación:}
\begin{itemize}
  \item Se realiza un unión (\texttt{JOIN}) entre la tabla \texttt{Producto} y \texttt{Categoria}.
  \item Solo se incluyen productos disponibles, es decir, con \texttt{stock > 0}.
  \item El ordenamiento se realiza primero por nombre de categoría y luego por precio.
\end{itemize}

\subsection{Clientes con pedidos pendientes y total de compras}

Este procedimiento muestra los clientes que tienen pedidos en estado pendiente, junto con el total de pedidos que han realizado.

\begin{lstlisting}
DELIMITER //

CREATE PROCEDURE PedidosPendientes()
BEGIN
  SELECT 
    cl.id_cliente,
    cl.nombre,
    cl.correo,
    COUNT(CASE WHEN p.estado = 'pendiente' THEN 1 END) AS pedidos_pendientes,
    COUNT(p.id_pedido) AS total_pedidos
  FROM Cliente cl
  LEFT JOIN Pedido p ON cl.id_cliente = p.id_cliente
  GROUP BY cl.id_cliente, cl.nombre, cl.correo
  HAVING pedidos_pendientes > 0;
END;
//

DELIMITER ;
\end{lstlisting}

\textbf{Explicación:}
\begin{itemize}
  \item Se realiza una \texttt{LEFT JOIN} entre clientes y sus pedidos.
  \item Se usa una condición \texttt{CASE WHEN} para contar únicamente los pedidos pendientes.
  \item Se utiliza \texttt{HAVING} para filtrar y mostrar solo los clientes con al menos un pedido pendiente.
\end{itemize}

\subsection{Top 5 productos con mejor calificación promedio}

Este procedimiento obtiene los cinco productos mejor calificados según la media de reseñas dadas por los clientes.

\begin{lstlisting}
DELIMITER //

CREATE PROCEDURE TopProductos()
BEGIN
  SELECT 
    p.nombre AS producto,
    ROUND(AVG(r.calificacion), 2) AS promedio_calificacion,
    COUNT(r.id_resena) AS total_resenas
  FROM Producto p
  JOIN Resena r ON p.id_producto = r.id_producto
  GROUP BY p.id_producto, p.nombre
  HAVING COUNT(r.id_resena) > 0
  ORDER BY promedio_calificacion DESC
  LIMIT 5;
END;
//

DELIMITER;
\end{lstlisting}

\textbf{Explicación:}
\begin{itemize}
  \item Se calcula el promedio de calificaciones para cada producto.
  \item Se filtran productos que no tienen reseñas.
  \item Se ordena el resultado en orden descendente por calificación y se limita a los primeros cinco.
\end{itemize}

\newpage


\section{Procedimientos almacenados en MySQL}
En esta sección se presentan los procedimientos almacenados implementados para gestionar operaciones fundamentales del sistema. Se incluyen:

\begin{enumerate}
  \item \textbf{Registrar un nuevo pedido:} Verifica el límite de 5 pedidos pendientes por cliente y que haya stock suficiente. \emph{Nota:} En este procedimiento se actualiza el stock de los productos, por lo que no es necesario un procedimiento adicional para la actualización del stock.
  \item \textbf{Registrar una reseña:} Permite insertar una reseña, delegando la validación a un trigger ya creado. \emph{Nota:} La validación se realiza en el trigger que se encuentra en la sección \textbf{Implementación de índices y triggers en MySQL, Sub sección: ``Trigger: validar reseñas solo de clientes que hayan comprado el producto''}.
  \item \textbf{Cambiar el estado de un pedido:} Permite actualizar de forma segura el estado de un pedido (por ejemplo, de pendiente a enviado).
\end{enumerate}

\subsection{1. Registrar un nuevo pedido}

El siguiente procedimiento almacena un nuevo pedido, verificando que el cliente no tenga ya 5 pedidos pendientes y que exista stock suficiente para cada producto incluido. Se utiliza un parámetro JSON para enviar los detalles del pedido (lista de productos y cantidades).

\begin{lstlisting}
DELIMITER //

CREATE PROCEDURE RegistrarPedido(
  IN p_id_cliente INT,
  IN p_fecha DATE,
  IN p_detalles JSON -- Ej: [{"id_producto":1, "cantidad":2}, {"id_producto":3, "cantidad":1}]
)
BEGIN
  DECLARE pedidos_pend INT;
  DECLARE stock_actual INT;
  DECLARE i INT DEFAULT 0;
  DECLARE detalles_count INT;
  DECLARE id_pedido_nuevo INT;
  DECLARE v_id_producto INT;
  DECLARE v_cantidad INT;

  -- Verificar que el cliente tenga menos de 5 pedidos pendientes
  SELECT COUNT(*) INTO pedidos_pend
  FROM Pedido
  WHERE id_cliente = p_id_cliente AND estado = 'pendiente';

  IF pedidos_pend >= 5 THEN
    SIGNAL SQLSTATE '45000'
    SET MESSAGE_TEXT = 'El cliente ya tiene 5 pedidos pendientes.';
  END IF;

  -- Crear el pedido
  INSERT INTO Pedido(fecha, estado, id_cliente)
  VALUES (p_fecha, 'pendiente', p_id_cliente);

  SET id_pedido_nuevo = LAST_INSERT_ID();
  SET detalles_count = JSON_LENGTH(p_detalles);

  -- Procesar cada detalle del pedido
  WHILE i < detalles_count DO
    SET v_id_producto = JSON_UNQUOTE(JSON_EXTRACT(p_detalles, CONCAT('$[', i, '].id_producto')));
    SET v_cantidad = JSON_UNQUOTE(JSON_EXTRACT(p_detalles, CONCAT('$[', i, '].cantidad')));

    -- Verificar stock disponible
    SELECT stock INTO stock_actual FROM Producto WHERE id_producto = v_id_producto;
    IF stock_actual < v_cantidad THEN
      SIGNAL SQLSTATE '45000'
      SET MESSAGE_TEXT = CONCAT('Stock insuficiente para el producto ID: ', v_id_producto);
    END IF;

    -- Insertar detalle del pedido
    INSERT INTO DetallePedido(id_pedido, id_producto, cantidad, precio_unitario)
    SELECT id_pedido_nuevo, v_id_producto, v_cantidad, precio
    FROM Producto
    WHERE id_producto = v_id_producto;

    -- Actualizar el stock (actualizacion incluida en este procedimiento)
    UPDATE Producto
    SET stock = stock - v_cantidad
    WHERE id_producto = v_id_producto;

    SET i = i + 1;
  END WHILE;
END;
//

DELIMITER ;
\end{lstlisting}

\subsection{2. Registrar una resena}

Este procedimiento inserta una resena para un producto. La validacion de que el cliente haya comprado el producto se realiza mediante un trigger (consulta la seccion Implementacion de indices y triggers en MySQL).

\begin{lstlisting}
DELIMITER //

CREATE PROCEDURE RegistrarResena(
  IN p_id_cliente INT,
  IN p_id_producto INT,
  IN p_calificacion INT,
  IN p_comentario TEXT
)
BEGIN
 
INSERT INTO Resena(calificacion, comentario, id_producto, id_cliente)
VALUES (p_calificacion, p_comentario, p_id_producto, p_id_cliente);
END;
//

DELIMITER ;
\end{lstlisting}

\textbf{Nota:} El trigger utilizado para validar que el cliente haya comprado el producto se encuentra en la secci\'on \textbf{Implementaci\'on de \'indices y triggers en MySQL}, en la subsecci\'on \emph{``Trigger: validar rese\~nas solo de clientes que hayan comprado el producto''}.

\subsection{3. Cambiar el estado de un pedido}

Este procedimiento permite actualizar el estado de un pedido existente de forma segura.

\begin{lstlisting}
DELIMITER //

CREATE PROCEDURE CambiarEstadoPedido(
  IN p_id_pedido INT,
  IN p_nuevo_estado ENUM('pendiente', 'enviado', 'entregado')
)
BEGIN
  DECLARE pedido_existente INT;

  -- Verificar que el pedido existe
  SELECT COUNT(*) INTO pedido_existente
  FROM Pedido
  WHERE id_pedido = p_id_pedido;

  IF pedido_existente = 0 THEN
    SIGNAL SQLSTATE '45000'
    SET MESSAGE_TEXT = 'El pedido no existe.';
  END IF;

  -- Actualizar el estado del pedido
  UPDATE Pedido
  SET estado = p_nuevo_estado
  WHERE id_pedido = p_id_pedido;
END;
//

DELIMITER ;
\end{lstlisting}

\textbf{Explicación general:}

\begin{itemize}
  \item En el procedimiento \textbf{RegistrarPedido} se verifica previamente que el cliente no exceda el límite de 5 pedidos pendientes y que cada producto cuente con stock suficiente. Se procesan los detalles del pedido enviados en formato JSON, y se actualiza el stock de cada producto al momento de generar el pedido.
  \item En el procedimiento \textbf{RegistrarResena}, la validación que asegura que el cliente haya comprado el producto se delega al trigger implementado, eliminando así redundancias en la lógica.
  \item En el procedimiento \textbf{CambiarEstadoPedido}, se valida la existencia del pedido antes de actualizar su estado, garantizando la integridad de la operación.
\end{itemize}






\end{document}