

Esta sección presenta una segunda tanda de procedimientos almacenados que permiten realizar operaciones comunes en la tienda en línea, desde mantenimiento de productos hasta reportes y validaciones adicionales.

\subsection{Eliminar reseñas de un producto y actualizar promedio}
\begin{lstlisting}[language=SQL]
DELIMITER //

CREATE PROCEDURE EliminarResenasYActualizarPromedio(
  IN p_id_producto INT
)
BEGIN
  DELETE FROM Resena
  WHERE id_producto = p_id_producto;
END;
//

DELIMITER ;
\end{lstlisting}
\textbf{Explicaci\'on:} Este procedimiento elimina todas las resenas asociadas a un producto y, si se lleva el promedio en una columna, se puede actualizar a NULL para indicar que ya no tiene calificaciones.

\subsection{Agregar producto evitando duplicados por nombre y categor\'ia}
\begin{lstlisting}[language=SQL]
DELIMITER //

CREATE PROCEDURE AgregarProductoSinDuplicado(
  IN p_nombre VARCHAR(100),
  IN p_descripcion TEXT,
  IN p_precio DECIMAL(10,2),
  IN p_stock INT,
  IN p_id_categoria INT
)
BEGIN
  DECLARE existe INT;

  SELECT COUNT(*) INTO existe
  FROM Producto
  WHERE nombre = p_nombre AND id_categoria = p_id_categoria;

  IF existe > 0 THEN
    SIGNAL SQLSTATE '45000'
    SET MESSAGE_TEXT = 'Ya existe un producto con ese nombre en esta categor\u00eda.';
  ELSE
    INSERT INTO Producto(nombre, descripcion, precio, stock, id_categoria)
    VALUES (p_nombre, p_descripcion, p_precio, p_stock, p_id_categoria);
  END IF;
END;
//

DELIMITER ;
\end{lstlisting}
\textbf{Explicaci\'on:} Asegura que no se repitan productos con el mismo nombre dentro de una misma categor\'ia, arrojando un error si ya existe.

\subsection{Actualizar direccion y telefono de un cliente}
\begin{lstlisting}[language=SQL]
DELIMITER //

CREATE PROCEDURE ActualizarDatosCliente(
  IN p_id_cliente INT,
  IN p_nueva_direccion TEXT,
  IN p_nuevo_telefono VARCHAR(20)
)
BEGIN
  DECLARE existe INT;

  SELECT COUNT(*) INTO existe
  FROM Cliente
  WHERE id_cliente = p_id_cliente;

  IF existe = 0 THEN
    SIGNAL SQLSTATE '45000'
    SET MESSAGE_TEXT = 'El cliente no existe.';
  ELSE
    UPDATE Cliente
    SET direccion = p_nueva_direccion,
        telefono = p_nuevo_telefono
    WHERE id_cliente = p_id_cliente;
  END IF;
END;
//

DELIMITER ;
\end{lstlisting}
\textbf{Explicaci\'on:} Permite modificar de forma segura los datos de contacto de un cliente, asegurando que el cliente exista antes de aplicar los cambios.

\subsection{Reporte de productos con bajo stock}
\begin{lstlisting}[language=SQL]
DELIMITER //

CREATE PROCEDURE ReporteStockBajo()
BEGIN
  SELECT id_producto, nombre, stock
  FROM Producto
  WHERE stock < 5
  ORDER BY stock ASC;
END;
//

DELIMITER ;
\end{lstlisting}
\textbf{Explicaci\'on:} Devuelve un listado de productos con stock inferior a 5 unidades, ordenado de menor a mayor.

\bigskip
Estos procedimientos mejoran la robustez del sistema al controlar integridad de datos, simplificar operaciones y facilitar consultas comunes para mantenimiento y gesti\'on.
